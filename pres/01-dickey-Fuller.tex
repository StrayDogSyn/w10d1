% Created 2025-06-16 Mon 13:25
% Intended LaTeX compiler: pdflatex
\documentclass[aspectratio=169]{beamer}
\usepackage[utf8]{inputenc}
\usepackage[T1]{fontenc}
\usepackage{graphicx}
\usepackage{longtable}
\usepackage{wrapfig}
\usepackage{rotating}
\usepackage[normalem]{ulem}
\usepackage{amsmath}
\usepackage{amssymb}
\usepackage{capt-of}
\usepackage{hyperref}
\usepackage[style=apa, backend=biber]{biblatex}
\DeclareLanguageMapping{american}{american-apa}
\addbibresource{./refs/refs.bib}
\AtEveryBibitem{\clearfield{note}}
\usepackage{./jtc}
\usetheme{default}
\author{Evan Misshula}
\date{\today}
\title{The Dickey-Fuller Test for Stationarity}
\hypersetup{
 pdfauthor={Evan Misshula},
 pdftitle={The Dickey-Fuller Test for Stationarity},
 pdfkeywords={},
 pdfsubject={},
 pdfcreator={Emacs 29.3 (Org mode 9.6.15)}, 
 pdflang={English}}
\begin{document}

\maketitle

\section{Stationarity}
\label{sec:orga32723b}
\begin{frame}[label={sec:org95b1c6e}]{1. What is Stationarity?}
\begin{itemize}
\item A time series \(\{y_t\}\) is \alert{\alert{strictly stationary}} if its statistical properties (like mean, variance, autocorrelation) do not change over time.
\item \alert{\alert{Weak (covariance) stationarity}}:  
\begin{enumerate}
\item \(E[y_t] = \mu\) (constant mean)
\item \(\mathrm{Var}(y_t) = \sigma^2\) (constant variance)
\item \(\mathrm{Cov}(y_t, y_{t+k})\) depends only on lag \(k\), not on \(t\).
\end{enumerate}
\end{itemize}
\end{frame}

\begin{frame}[label={sec:orgf501679}]{2. What is a Unit Root?}
\begin{itemize}
\item Consider the AR(1) model: -\(y_t = \phi\,y_{t-1} + \varepsilon_t\), where \(\varepsilon_t \sim \mathrm{IID}(0, \sigma^2)\).
\item If \(|\phi| < 1\), the process is stationary.
\item If \(\phi = 1\), there is a \alert{unit root}, and \(y_t =
    y_{t-1} + \varepsilon_t\) is a \alert{random walk} \(\rightarrow\) non‑stationary.
\end{itemize}
\end{frame}

\begin{frame}[label={sec:org556f11d}]{What does IID mean?}
\begin{itemize}
\item \alert{IID} stands for \alert{independent and identically distributed}
\begin{itemize}
\item Each \(\varepsilon_t\) is drawn from the same distribution
(e.g., Normal with mean 0, variance \(\sigma^2\))
\item No dependence across time: each error term is independent of the
others
\end{itemize}
\end{itemize}
\end{frame}

\begin{frame}[label={sec:org882d165}]{3. Rewriting Using Differences}
\begin{itemize}
\item Define the first difference:  \(\Delta y_t = y_t - y_{t-1}\).
\item Then:  
\begin{itemize}
\item \(\Delta y_t = (\phi - 1)\,y_{t-1} + \varepsilon_t =
      \delta\,y_{t-1} + \varepsilon_t\),
\end{itemize}
where \(\delta = \phi - 1\).
\item Hypotheses:  
\begin{itemize}
\item \(H_0\): \(\delta = 0\) (i.e., \(\phi = 1\),
non‑stationary).
\item \(H_1\): \(\delta < 0\) (i.e., \(|\phi| < 1\),
stationary).
\end{itemize}
\end{itemize}
\end{frame}

\section{Tests}
\label{sec:org30a5550}
\begin{frame}[label={sec:orgc863dd1}]{4. Dickey-Fuller Test Statistic}
\begin{itemize}
\item Estimate regression: \(\Delta y_t = \delta\,y_{t-1} +
    \varepsilon_t\).

\item Compute t‑statistic: \(t_\delta = \widehat\delta / \mathrm{SE}(\widehat\delta)\).
\item Compare \(t_\delta\) to non‑standard critical values (e.g., from
MacKinnon tables).
\end{itemize}
\end{frame}

\begin{frame}[label={sec:org6efc78e}]{5. Augmented Dickey-Fuller (ADF)}
\begin{itemize}
\item Add intercept and trend if needed:  
\[
    \Delta y_t = \alpha + \beta\,t + \delta\,y_{t-1} + \sum_{i=1}^p \gamma_i\,\Delta y_{t-i} + \varepsilon_t
    \].
\item The additional lag terms control for higher‑order correlation in
residuals.
\end{itemize}
\end{frame}
\end{document}
