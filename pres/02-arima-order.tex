% Created 2025-06-16 Mon 13:08
% Intended LaTeX compiler: pdflatex
\documentclass[aspectratio=169]{beamer}
\usepackage[utf8]{inputenc}
\usepackage[T1]{fontenc}
\usepackage{graphicx}
\usepackage{longtable}
\usepackage{wrapfig}
\usepackage{rotating}
\usepackage[normalem]{ulem}
\usepackage{amsmath}
\usepackage{amssymb}
\usepackage{capt-of}
\usepackage{hyperref}
\usepackage[style=apa, backend=biber]{biblatex}
\DeclareLanguageMapping{american}{american-apa}
\addbibresource{./refs/refs.bib}
\AtEveryBibitem{\clearfield{note}}
\usepackage{./jtc}
\usetheme{default}
\author{Evan Misshula}
\date{\today}
\title{Arima order and seasonality}
\hypersetup{
 pdfauthor={Evan Misshula},
 pdftitle={Arima order and seasonality},
 pdfkeywords={},
 pdfsubject={},
 pdfcreator={Emacs 29.3 (Org mode 9.6.15)}, 
 pdflang={English}}
\begin{document}

\maketitle

\section{Review}
\label{sec:org7db1a04}
\begin{frame}[label={sec:org950063c}]{Introduction to ARIMA Models}
\begin{itemize}
\item ARIMA stands for:
\begin{itemize}
\item AR: Autoregressive
\item I: Integrated (differencing)
\item MA: Moving Average
\end{itemize}
\item ARIMA models are used for univariate time series forecasting.
\item General form: ARIMA(p, d, q)
\end{itemize}
\end{frame}



\begin{frame}[label={sec:org7246426}]{Autoregressive (AR) Model}
\begin{itemize}
\item An AR(p) model expresses current value \(y_t\) as a linear combination of past values:
\[
    y_t = \phi_1 y_{t-1} + \phi_2 y_{t-2} + \cdots + \phi_p y_{t-p} + \varepsilon_t
  \]
where \(\varepsilon_t \sim \text{IID}(0, \sigma^2)\)
\end{itemize}
\end{frame}

\begin{frame}[label={sec:org62a8983}]{Moving Average (MA) Model}
\begin{itemize}
\item An MA(q) model expresses \(y_t\) as a linear combination of past error terms:
\[
    y_t = \mu + \varepsilon_t + \theta_1 \varepsilon_{t-1} + \cdots + \theta_q \varepsilon_{t-q}
  \]
where \(\varepsilon_t\) is white noise
\end{itemize}
\end{frame}

\section{ACF and PACF}
\label{sec:orgd3ea40f}

\begin{frame}[label={sec:org6d7a24b}]{Autocorrelation Function (ACF)}
\begin{itemize}
\item Measures linear correlation between \(y_t\) and \(y_{t-k}\)
\[
    \rho_k = \frac{\text{Cov}(y_t, y_{t-k})}{\sqrt{\text{Var}(y_t)\, \text{Var}(y_{t-k})}}
  \]
\end{itemize}
\end{frame}

\begin{frame}[label={sec:org93cd145}]{Partial Autocorrelation Function (PACF)}
\begin{itemize}
\item Measures correlation between \(y_t\) and \(y_{t-k}\), controlling for \(y_{t-1}, \ldots, y_{t-k+1}\)
\item Formally, PACF at lag k is the coefficient \(\phi_{kk}\) in the AR(k) model
\end{itemize}
\end{frame}

\section{Identifying ARIMA Order}
\label{sec:org5d5afea}

\begin{frame}[label={sec:org86c9185}]{How to choose p, d, q?}
We use:

\begin{itemize}
\item \alert{Differencing}: determines \(d\), the number of times to difference the data to make it stationary
\item \alert{ACF plot}: helps choose \(q\), the MA order
\item \alert{PACF plot}: helps choose \(p\), the AR order
\end{itemize}
\end{frame}

\begin{frame}[label={sec:org28253ba}]{Rules of Thumb}
\begin{itemize}
\item If ACF cuts off sharply after lag \(q\): likely MA(q)
\item If PACF cuts off sharply after lag \(p\): likely AR(p)
\item If both ACF and PACF decay gradually: try ARMA(p,q)
\end{itemize}
\end{frame}

\section{Python Implementation}
\label{sec:orga24e08e}
\begin{frame}[label={sec:org6d40dc5}]{Let's run code}
07-arima-order.py
\end{frame}


\section{Summary}
\label{sec:org0b14245}
\begin{frame}[label={sec:org96a5939},fragile]{Summary of S-Arima}
 \begin{itemize}
\item ARIMA models handle trend and autocorrelation
\item Use differencing + ACF + PACF to identify parameters
\item Fit models with \texttt{`statsmodels`} in Python
\end{itemize}

\begin{block}{References}
\begin{itemize}
\item Box, Jenkins, Reinsel: \alert{Time Series Analysis}
\item Statsmodels documentation: \texttt{ARIMA} and \texttt{plot\_acf/pacf}
\end{itemize}
\end{block}
\end{frame}
\end{document}
